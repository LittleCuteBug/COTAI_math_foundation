%%% Template originaly created by Karol Kozioł (mail@karol-koziol.net) and modified for ShareLaTeX use

\documentclass[a4paper,11pt]{article}

\usepackage[T1]{fontenc}
\usepackage[utf8]{inputenc}
\usepackage[utf8]{vietnam}
\usepackage{graphicx}
\usepackage{xcolor}
\renewcommand\familydefault{\sfdefault}
\usepackage{tgheros}
\usepackage[defaultmono]{droidmono}
\usepackage{amsmath,amssymb,amsthm,textcomp}
\usepackage{enumerate}
\usepackage{multicol}
\usepackage{tikz}
\usepackage{geometry}
\geometry{left=25mm,right=25mm,%
bindingoffset=0mm, top=20mm,bottom=20mm}
\linespread{1.3}
\newcommand{\linia}{\rule{\linewidth}{0.5pt}}
% custom theorems if needed
\newtheoremstyle{mytheor}
    {1ex}{1ex}{\normalfont}{0pt}{\scshape}{.}{1ex}
    {{\thmname{#1 }}{\thmnumber{#2}}{\thmnote{ (#3)}}}
\theoremstyle{mytheor}
\newtheorem{defi}{Definition}
% my own titles
\makeatletter
\renewcommand{\maketitle}{
\begin{center}
\vspace{2ex}
{\huge \textsc{\@title}}
\vspace{1ex}
\\
\linia\\
\@author \hfill \@date
\vspace{4ex}
\end{center}
}
\makeatother
%%%

% custom footers and headers
\usepackage{fancyhdr}
\pagestyle{fancy}
\lhead{}
\chead{}
\rhead{}
\lfoot{}
\cfoot{}
\rfoot{Page \thepage}
\renewcommand{\headrulewidth}{0pt}
\renewcommand{\footrulewidth}{0pt}
%

% code listing settings
\usepackage{listings}
\lstset{
    language=Python,
    basicstyle=\ttfamily\small,
    aboveskip={1.0\baselineskip},
    belowskip={1.0\baselineskip},
    columns=fixed,
    extendedchars=true,
    breaklines=true,
    tabsize=4,
    prebreak=\raisebox{0ex}[0ex][0ex]{\ensuremath{\hookleftarrow}},
    frame=lines,
    showtabs=false,
    showspaces=false,
    showstringspaces=false,
    keywordstyle=\color[rgb]{0.627,0.126,0.941},
    commentstyle=\color[rgb]{0.133,0.545,0.133},
    stringstyle=\color[rgb]{01,0,0},
    numbers=left,
    numberstyle=\small,
    stepnumber=1,
    numbersep=10pt,
    captionpos=t,
    escapeinside={\%*}{*)}
}
\usepackage{hyperref}
\hypersetup{
    colorlinks=true,
    linkcolor=blue,
    filecolor=magenta,      
    urlcolor=cyan,
}
\usepackage{amsmath}
\makeatletter
\renewcommand*\env@matrix[1][*\c@MaxMatrixCols c]{%
  \hskip -\arraycolsep
  \let\@ifnextchar\new@ifnextchar
  \array{#1}}
\makeatother
%%%----------%%%----------%%%----------%%%----------%%%

\begin{document}

\title{Lecture 2 Assignment Extra}

\author{Nguyen Quang Huy}

\date{07/04/2020}

\maketitle

\section*{P2.5}

Với ma trận $A_{m*n}$ và ma trận đường chéo $P_{n*n}$ có đường chéo bằng $\{p_1,p_2,...p_n\}$ ta có:
\begin{align}
    \nonumber A*P &= \begin{bmatrix} a_{11} & a_{12} & \hdots & a_{1n} \\ 
                        a_{21} & a_{22} & \hdots & a_{2n} \\
                        \vdots & \vdots & \ddots & \vdots \\
                        a_{m1} & a_{m2} & \hdots & a_{mn}
                        \end{bmatrix}
        * \begin{bmatrix} p_1 & 0 & \hdots & 0\\
                        0 & p_2 & \hdots & 0 \\
                        \vdots & \vdots & \ddots & \vdots \\
                        0 & 0 & \hdots & p_n
                        \end{bmatrix}
        \\
    \nonumber     &= \begin{bmatrix} a_{11}*p_1 & a_{12}*p_2 & \hdots & a_{1n}*p_n \\ 
                        a_{21}*p_1 & a_{22}*p_2 & \hdots & a_{2n}*p_n \\
                        \vdots & \vdots & \ddots & \vdots \\
                        a_{m1}*p_1 & a_{m2}*p_2 & \hdots & a_{mn}*p_n
                        \end{bmatrix}
\end{align}
Tương tự, với ma trận $A_{m*n}$ và ma trận đường chéo $P_{m*m}$ có đường chéo bằng $\{p_1,p_2,...p_m\}$ ta có:
\begin{align}
    \nonumber P*A &= \begin{bmatrix} p_1 & 0 & \hdots & 0\\
                        0 & p_2 & \hdots & 0 \\
                        \vdots & \vdots & \ddots & \vdots \\
                        0 & 0 & \hdots & p_m
                        \end{bmatrix}
        * \begin{bmatrix} a_{11} & a_{12} & \hdots & a_{1n} \\ 
                        a_{21} & a_{22} & \hdots & a_{2n} \\
                        \vdots & \vdots & \ddots & \vdots \\
                        a_{m1} & a_{m2} & \hdots & a_{mn}
                        \end{bmatrix}
        \\
    \nonumber     &= \begin{bmatrix} a_{11}*p_1 & a_{12}*p_1 & \hdots & a_{1n}*p_1 \\ 
                        a_{21}*p_2 & a_{22}*p_2 & \hdots & a_{2n}*p_2 \\
                        \vdots & \vdots & \ddots & \vdots \\
                        a_{m1}*p_m & a_{m2}*p_m & \hdots & a_{mn}*p_m
                        \end{bmatrix}
\end{align}
Nhận xét: phép nhân ma trận $A_{m*n}*P_{n*n}$ scale mỗi cột của ma trận A với hệ số $p_i$, phép nhân ma trận $P_{m*m}*A_{m*n}$ scale mỗi hàng của ma trận A với hệ số $p_i$
\section*{P2.6}
Với $A \in R^{m*n}, B \in R^{n*p}$, ta có:
\begin{align}
    \nonumber AB &= \begin{matrix}\\ (i_1) \\ (i_2) \end{matrix}\begin{matrix}\begin{matrix}(j_1)&(j_2) \end{matrix}\\
            \begin{bmatrix} [c|c]
            A_{11} & A_{12}\\ \hline
            A_{21} & A_{22}
            \end{bmatrix} \end{matrix}
        *\begin{matrix}\\ (k_1) \\ (k_2) \end{matrix}\begin{matrix}\begin{matrix}(l_1)&(l_2) \end{matrix}\\
            \begin{bmatrix} [c|c]
            B_{11} & B_{12}\\ \hline
            B_{21} & B_{22}
            \end{bmatrix} \end{matrix}\\
    \nonumber    &= 
    \begin{matrix} \\ (i_1) \\ (i_2) \end{matrix}
    \begin{matrix}
            \begin{matrix}
                (l_1) & & & & & & & & & (l_2)
            \end{matrix}\\
            \begin{bmatrix} [c|c]
            A_{11}*B_{11}+A_{12}*B_{21} & A_{11}*B_{12}+A_{12}*B_{22}\\ \hline
            A_{21}*B_{11}+A_{22}*B_{21} & A_{21}*B_{12}+A_{22}*B_{22}
            \end{bmatrix}
    \end{matrix}
\end{align}
\textbf{Nhận xét:}\\
Để thực hiện phép nhân ma trận, các kích thước của các ma trận con phải thỏa mãn:
\begin{itemize}
    \item $i_1+i_2=m$, $j_1+j_2=n$
    \item $k_1+k_2=n$, $l_1+l_2=p$
    \item $j_1=k_1$, $j_2=k_2$
\end{itemize}
\section*{P2.7}
Trong không gian 2D:
$$
    \nonumber ReflectOrigin()*Translate(20,10)*Rotate(30^o) = \begin{bmatrix}
        -cos(30^o) & -sin(30^o) & -20 \\
        sin(30^o) & -cos(30^o) & -10 \\
        0 & 0 & 1
    \end{bmatrix}
$$
Trong không gian 3D:
$$
    \nonumber ReflectOrigin()*Translate(20,10,0)*RotateZ(30^o) = \begin{bmatrix}
        -cos(30^o) & -sin(30^o) & 0 & -20 \\
        sin(30^o) & -cos(30^o) & 0 & -10 \\
        0 & 0 & 1 & 0 \\
        0 & 0 & 0 & 1 
    \end{bmatrix}
$$
\href{https://colab.research.google.com/drive/11goRiQbw1PLvmmVFmF0eqNn7gru4-Ozw}{Code link 
here}\\
\href{https://drive.google.com/open?id=1IOk26Iip__HIKyEfTlNXBgDS0Sd6v_dP}{Gif P2.7.1}\\
\href{https://drive.google.com/open?id=1c8WvxVmPRdflaSU3IE3rp4fnA8AFg8kR}{Gif P2.7.2}
\section*{P2.8}
Để giảm chiều dữ liệu: $\Bar{z} \xrightarrow{A} z = (1,w_1,w_2,..w_k)^T$ thì ma trận $A_{(k+1)*(n+1)}$ có dạng
$
    A = \begin{bmatrix}[c|c]
        \bar{I}_{(k+1)*(k+1)} & \bar{0}_{(k+1)*(n-k)}
    \end{bmatrix}
$
$$
    A*z = \begin{bmatrix} [cccc|ccc]
    1 & 0 & \hdots & 0 & 0 & \hdots & 0\\
    0 & 1 & \hdots & 0 & 0 & \hdots & 0\\
    \vdots & \vdots & \ddots & \vdots & \vdots & \ddots & \vdots \\
    0 & 0 & \hdots & 1 & 0 & \hdots & 0
    \end{bmatrix}
    * 
    \begin{bmatrix}
    1\\w_1\\w_2\\ \vdots \\ w_n
    \end{bmatrix} = 
    \begin{bmatrix}
    1\\w_1\\w_2\\ \vdots \\ w_k
    \end{bmatrix}
$$
\section*{P2.9}
Xét ma trận $A \in R^{m*n}$:
Với mọi x thuộc Ker(A), ta có $ A*x = \bar{0}\\
    \xrightarrow{} A^T*A*x = \bar{0}\\
    \xrightarrow{} x \in Ker(A^T*A) \hfill \textit{(1)} \\
$
Với mọi x thuộc $Ker(A^T*A)$, ta có $ A^T*A*x = \bar{0}\\
    \xrightarrow{} x^T*A^T*A*x = \bar{0}\\
    \xrightarrow{} (A*x)^T*A*x = \bar{0}\\
    \xrightarrow{} A*x = \bar{0}\\
    \xrightarrow{} x \in Ker(A) \hfill \textit{(2)}\\ 
$
Từ (1) và (2) ta có $Ker(A^T*A) = Ker(A) \\ 
\xrightarrow{} Nullity(A^T*A) = Nullity(A)$\\
Mặt khác ta có $A \in R^{m*n}$ và $A^T*A \in R^{n*n} \\ 
\xrightarrow{} n - Nullity(A^T*A) = n - Nullity(A)$ hay $Rank(A^T*A) = Rank(A)$\\
Chứng minh tương tự ta có $Rank(A*A^T) = Rank(A^T)$\\
Theo kết quả P2.11: $Rank(A) = Rank(A^T)$\\
$\xrightarrow{} Rank (A^T*A) = Rank(A) = Rank(A^T) = Rank(A*A^T)$

\section*{P2.10}
\subsection*{Chứng minh $(A*B*C)^{-1}=C^{-1}*B^{-1}*A^{-1}$}
Ta có $(A*B*C)^{-1}*(A*B*C) = I$ và $C^{-1}*B^{-1}*A^{-1}*A*B*C = I$\\
Suy ra $(A*B*C)^{-1}*(A*B*C) = C^{-1}*B^{-1}*A^{-1}*A*B*C$\\
$\xrightarrow{}(A*B*C)^{-1}*A*B*C*C^{-1}*B^{-1}*A^{-1} = C^{-1}*B^{-1}*A^{-1}*A*B*C*C^{-1}*B^{-1}*A^{-1}$\\
$\xrightarrow{}(A*B*C)^{-1}=C^{-1}*B^{-1}*A^{-1}$
\subsection*{Chứng minh $A^{-1^T} = A^{T^{-1}} := A^{-T}$}
Ta có: $A^{T^{-1}}*A^T=I$ và $A^{-1^T}*A^T= (A*A^{-1})^T = I^T = I$ \\
$\xrightarrow{}A^{T^{-1}}*A^T = A^{-1^T}*A^T$ \\
$\xrightarrow{}A^{T^{-1}}*A^T*A^{T^{-1}} = A^{-1^T}*A^T*A^{T^{-1}}$ \\
$\xrightarrow{}A^{T^{-1}} = A^{-1^T}$ 

\section*{P2.11}
\subsection*{Chứng minh $Rank(A)<min(m,n)$}
Xét ma trận $A_{m*n}$, ta có $Rank(A) = Dim(img(A))$ với $img(A) = \{x|A*x!=\bar{0},x \in R^n\}$. \\
Do $img(A) \subseteq R^n \xrightarrow{} Rank(A) <= n \hfill \textit{(1)}$\\ 
Do $img(A) = span(a_1,a_2,..a_m)$ với $a_i$ là hàng i của A $\xrightarrow{} Rank(A) <= m\hfill \textit{(2)}$\\
Từ (1) và (2) $\xrightarrow{} Rank(A)<min(m,n)$\\
Tương tự cho  $Rank(A^T)<min(m,n)$
\subsection*{Chứng minh $Rank(A) = Rank(A^T)$}
Điều này tương đương với việc chứng minh$ dim(Col(A)) = dim(Row(A)$\\
Gọi $B = b_1,b_2,...b_p$ với $u_i \in R^m$ là basis vector của Col(A)\\
Ta có $A = B_{m*p}*W_{p*n}$. 
\\ Do mỗi hàng của A đều là tổ hợp tuyến tính của p vector là các hàng của W\\
$\xrightarrow{} Row(A)\subseteq Span(w_1,w_2..w_p)$ với $w_i$ là hàng i của W\\
$\xrightarrow{} Dim(Row(A)) <= Dim(Col(A))$\\
Tương tự $\xrightarrow{} Dim(Row(A)) >= Dim(Col(A))$\\
$\xrightarrow{} Dim(Row(A)) = Dim(Col(A))$
\end{document}
